\documentclass[a4paper]{article}

\usepackage{amsmath}
\usepackage{amsfonts}
\usepackage{amssymb}
\usepackage{graphicx}

\begin{document}

\Large
 \begin{center}
Simulating Pathogenic Behavior and Transmission in Populations\\ 

\hspace{10pt}

\large
Carter Hall$^1$, Taylor Beaman$^1$, Molly McDermott$^1$, Sam Orenstein$^1$\\

\hspace{10pt}

\small  
$^1$ Department of Biology, University of North Carolina at Chapel Hill\\

\end{center}

\hspace{10pt}

\normalsize

\hrule

\section{Abstract}

In a world recently held captive by the coronavirus (COVID-19) pandemic, an understanding of the population dynamics through which viruses propagate their existence is critical. Fortunately, the field of epidemiology has, for decades, dedicated itself to quantifying the effects of pandemics and epidemics in efforts to better advise public policymakers and average citizens as they live through unprecedented times.

This report looks to employ biological and statistical techniques to characterize pathogenic behavior across a myriad of scenarios. The topics discussed represent risks to the health of humans and include, but are not limited to, viral mutations, co-infection (simultaneous infection of multiple viral strains), and the efficacy and usefulness of vaccinations. Such analysis is achieved through temporal simulations of systems of equations analogous to the SIR family of models [\textbf{Reference to some website about SIR models}], as well as a spatiotemporal, Monte Carlo simulation that invokes principles of stochasticity to mimic population dynamics.

\begin{center}
\textbf{Key Words}: epidemiology, SIR Modeling, pathogen, population dynamics, Monte Carlo simulation 
\end{center}

\end{document}